\documentclass[document.tex]{subfiles}
\begin{document}
	\addcontentsline{toc}{chapter}{Abstract}
	\begin{center}
		\textbf{{\fontsize{16pt}{18}\selectfont Abstract}}
		\vspace{1cm}
	\end{center}
	Hyperspectral sensors are devices that acquire images over hundreds of spectral bands, thereby enabling the extraction of spectral signatures for objects or materials observed. It has wide range of applications in ground object detection, which makes hyperspectral image analysis an important field of research. Most informative features selection for effective hyperspectral image classification is a difficult task. Different approaches has been proposed to select only relevant features from those large correlated data set. The problem can be address using feature extraction and feature selection. Principal component analysis (PCA) is one of the most popular feature extraction technique, though its components is not always suitable for better classification accuracy. This research proposed a target class oriented features selection method which uses normalized mutual information (nMI) measure with two constraints to maximize general relevance and minimize redundancy on the components obtained via PCA. In this research the proposed feature mining approach is combined with kernel support vector machine (SVM) classifier for the effective classification object. Target class oriented features selection approach shows significant improvement in terms of classification accuracy 97.43\% of real hyperspectral data. A comparison among relevant and recent feature selection techniques in terms of their classification accuracy  is provided using hyperspectral image.

	\clearpage
	
\end{document}
